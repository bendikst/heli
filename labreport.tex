\documentclass[11pt, a4paper, USenglish]{article} % change ``USenglish'' to ``norsk'' if applicable.

\usepackage{kyblab} % Contains all included packages. See kyblab.sty.
\addbibresource{bibliography.bib} % Makes the bibliography file available to biblatex.

\begin{document}

% Titlepage
\title{\textbf{Helicopter lab report}\\ \textbf{TTK 4115}}
\author{Group 53\\Anders Granberg Drønnen 768686\\Bendik Steinkjer Standal 757944\\Filip Gornitzka Abelson 768687}
\date{October 2017}
\begin{titlepage}
    \maketitle
    \begin{figure}
    \centering
    \includegraphics[width=0.5\textwidth]{figures/itk_ntnu}\\
    Department of Engineering Cybernetics
    \end{figure}
    \thispagestyle{empty}
\end{titlepage}

% Abstract
\newpage
\input{abstract}
\thispagestyle{empty} % Avoid page numbering on the abstract page.

% TOC
\newpage
\tableofcontents
\thispagestyle{empty} % Avoid page numbering on the table of contents.

% Main content
\newpage
\setcounter{page}{1}
\include{intro}
\section{Part I - Mathematical modeling}\label{sec:part1}

\subsection{Problem 1}
\begin{subequations} \label{eq:motor}
    \begin{gather}
        F_f = K_f V_f \label{eq:motor_front} \\
        F_b = K_f V_b \label{eq:motor_back}
    \end{gather}
\end{subequations}
\begin{subequations} \label{eq:model}
    \begin{gather}
        J_p \ddot{p} = L_1 V_d \label{eq:model_pitch}\\
        J_e \ddot{e} = L_2 \cos(e) + L_3 V_s \cos(p) \label{eq:model_elev}\\
        J_{\lambda} \ddot{\lambda} = L_4 V_s \cos(e) \sin(p) \label{eq:model_travel}
    \end{gather}
\end{subequations}
By using Newton's second low for rotation, we can compute the equations of motion, in the form as stated above. We will insert equation \eqref{eq:motor_front} and \eqref{eq:motor_back}.
\begin{equation} \label{eq:newton2rot}
    J \alpha = \sum{\tau}
\end{equation}
We will begin with the equation for pitch. The torques generated by gravity cancel each other out, so we end up with the torques generated by the motors.
\begin{gather*}
    J_p \ddot{p} = l_p F_{g,b} - l_p F_{g,f} + l_p F_f - l_p F_b \\
    = l_p K_f V_f - l_p K_f V_b = l_p K_f (V_f - V_b) \\
    \implies J_p \ddot{p} = L_1 V_d
\end{gather*}
Now we will look at the equation for elevation. We must use basic trigonometry to find the right components of the forces. As we can see, the forces are dependant on both pitch angle and elevation angle.
\begin{gather*}
    J_p \ddot{e} = l_c F_{g,c} \cos(e) + l_h (F_f \cos(p) + F_b \cos(p) - l_h F_{g,f} \cos(e) - l_h F_{g,b} \cos(e) \\
    = l_c F_{g,f} \cos(e) + l_h K_f \cos(p) (V_f + V_b) - 2 l_h m_p g \cos(e) \\
    = g(l_c m_c - 2 l_h m_p) \cos(e) + l_h K_f V_s \cos(p) \\
    \implies J_e \ddot{e} = L_2 \cos(e) + L_3 V_s \cos(p)
\end{gather*}
The last equation is the one for travel. Once again, we must use basic trigonometry to find the force and lever arm which are orthogonal. 
\begin{gather*}
    J_\lambda \ddot{\lambda} = \tau_f - \tau_b \\
    = - l_h \cos(e) F_f \sin(p) - l_h \cos(e) F_b \sin(p)\\
    = - l_h K_f (V_f + V_b) \cos(e) \sin(p) = - l_h K_f V_s \cos(e) \sin(p)\\
    \implies J_\lambda \ddot{\lambda} = L_4 V_s \cos(e) \sin(p)
\end{gather*}
Now we have expressions for the constants $L_{1-4}$:
\begin{subequations}\label{eq:constants-L}
    \begin{align}
        & L_1 = l_p K_f \label{eq:constant-L1}\\
        & L_2 = g (l_c m_c - 2 l_h m_p) \label{eq:constant-L2}\\
        & L_3 = l_h K_f \label{eq:constant-L3} \\
        & L_4 = - L_3 = - l_h K_f \label{eq:constant-L4}
    \end{align}
\end{subequations}
\begin{subequations}\label{eq:motor_voltage}
    \begin{gather}
        V_s = V_f + V_b \label{motor_voltage_sum}\\
        V_d = V_f - V_b \label{motor_voltage_difference}
    \end{gather}
\end{subequations}
\subsection{Problem 2}\label{subsec:P1p2}
To be able to control our helicopter, we must linearize the equation of motion. We will linearize the model around the point $(p,e,\lambda)^T=(p^*,e^*,\lambda^*)^T$ with $p^*=e^*=\lambda^*=0$. Now we determine the voltages $V_s^*$ and $V_d^*$, such that for all time $\dot{p} = \dot{e} = \dot{\lambda} = 0$. 
This implies that $\ddot{p} = \ddot{e} = \ddot{\lambda} = 0$ for all time. \\
\\Equation \eqref{eq:model_pitch} yields;
\begin{align}
    J_p \cdot 0 = L_1 V_d \implies V_d^* = 0
\end{align}
Equation \eqref{eq:model_elev} yields;
\begin{align}
    J_e \cdot 0 &= L_2 \cos(0) + L_3 V_s^* \cos(0) \nonumber \\
    L_3 V_s^* &= - L_2 \implies V_s^* = -\frac{L_2}{L_3}\label{eq:vs_tilde}
\end{align}
We use the following coordinate transformation to linearize and rewrite the equations of motion. 
\begin{equation}\label{eq: coord_trans}
    \begin{bmatrix} 
        \tilde{p} \\ \tilde{e} \\ \tilde{\lambda}
    \end{bmatrix}
    =
    \begin{bmatrix} 
        p \\ e \\ \lambda
    \end{bmatrix}
    -
    \begin{bmatrix} 
        p^* \\ e^* \\ \lambda^*
    \end{bmatrix}
    \text{ and }
    \begin{bmatrix}
        \tilde{V_s} \\ \tilde{V_d}
    \end{bmatrix}
    =
    \begin{bmatrix}
        V_s \\ V_d
    \end{bmatrix}
    -
    \begin{bmatrix}
        V_s^* \\ V_d^*
    \end{bmatrix}
\end{equation}



\textbf{DE UNDER BØR KANSKJE SETTES ET ANNET STED??}
\begin{subequations} \label{eq:moments_inertia}
    \begin{align}
        & J_p = 2 m_p l_p^2 \\
        & J_e = m_c l_c^2 + 2 m_p l_h^2 \\
        & J_\lambda = m_c l_c^2 + 2 m_p (l_h^2 + l_p^2)
    \end{align}
\end{subequations}
\begin{subequations}
    \begin{align}
        & \ddot{\tilde{p}} = \frac{L_1}{J_p} \tilde{V_d} \label{eq:trans_model_pitch}\\
        & \ddot{\tilde{e}} = \frac{L_2}{J_e} \cos(\tilde{e}) + \frac{L_3}{J_e}(\tilde{V_s} - \frac{L_2}{L_3}) \cos(\tilde{p})\label{eq:trans_model_elev} \\
        & \ddot{\tilde{\lambda}} = \frac{L_4}{J_\lambda } (\tilde{V_s}-\frac{L_2}{L_3}) \cos(\tilde{e}) \sin(\tilde{p})\label{eq:trans_model_travel}
    \end{align}
\end{subequations}
We set up a state-space model on the following form and linearize the model around $\mathbf{x}_0 = \mathbf{0}$ and $\mathbf{u}_0 = (0, -\frac{L_2}{L_3})^T$.
\begin{equation*}
    \mathbf{\dot{x}} = \mathbf{Ax} + \mathbf{Bu}
\end{equation*}
Where, 
\begin{equation*}
    \mathbf{x} = 
    \begin{bmatrix}
        x_1 \\ x_2 \\ \vdots \\ x_6
    \end{bmatrix}
    =
    \begin{bmatrix}
        \tilde{p} \\ \dot{\tilde{p}} \\ \tilde{e} \\ \dot{\tilde{e}} \\ \tilde{\lambda} \\ \dot{\tilde{\lambda}} \\ 
    \end{bmatrix}
    \text{ and }
    \mathbf{u} =
    \begin{bmatrix}
        \tilde{u_1} \\ \tilde{u_2}
    \end{bmatrix}
    =
    \begin{bmatrix}
        \tilde{V_s} \\ \tilde{V_d}
    \end{bmatrix}
\end{equation*}
Using equation \eqref{eq:trans_model_pitch}, \eqref{eq:trans_model_elev} and \eqref{eq:trans_model_travel} we get the following expression for $\mathbf{\dot{x}}$.
\begin{equation*}
    \mathbf{\dot{x}} = 
    \begin{bmatrix}
        f_1 \\ f_2 \\ \vdots \\ f_6
    \end{bmatrix}
    =
    \begin{bmatrix}
        x_2 \\ \frac{L_1}{J_p}u_2 \\ \frac{L_2}{J_e}\cos(x_3) + \frac{L_3}{J_e} u_1 \cos(x_1) \\ x_6 \\ \frac{L_4}{J_\lambda}u_1 \cos(x_3) \sin(x_1)
    \end{bmatrix}
\end{equation*}
Now we can find the matrices \textbf{A} and \textbf{B}.
\begin{gather*}
    \mathbf{A} = \frac{\partial \mathbf{f}}{\partial \mathbf{x}} = 
    \left.
    \begin{bmatrix}
        \frac{\partial f_1}{\partial x_1} & \cdots &\frac{\partial f_1}{\partial x_6} \\
        \vdots & \ddots & \vdots \\
        \frac{\partial f_6}{\partial x_1} & \cdots & \frac{\partial f_6}{\partial x_6}
    \end{bmatrix}
    \right|_{\mathbf{x_0}, \mathbf{u_0}}
    \text{, }
    \mathbf{B} = \frac{\partial \mathbf{f}}{\partial \mathbf{u}} = 
    \left.
    \begin{bmatrix}
        \frac{\partial f_1}{\partial u_1} & \frac{\partial f_1}{\partial u_2} \\
        \frac{\partial f_6}{\partial u_1} & \frac{\partial f_6}{\partial u_2}
    \end{bmatrix}
    \right|_{\mathbf{x_0}, \mathbf{u_0}}
\end{gather*}
Calculating the partial derivatives and inserting for the points we are linearizing about, we get
\begin{gather}
    \mathbf{A} = 
    \begin{bmatrix}
        0 & 1 & 0 & 0 & 0 & 0 \\
        0 & 0 & 0 & 0 & 0 & 0 \\
        0 & 0 & 0 & 1 & 0 & 0 \\
        0 & 0 & 0 & 0 & 0 & 0 \\
        0 & 0 & 0 & 0 & 0 & 1 \\
        -\frac{L_2 L_4}{J_\lambda L_3} & 0 & 0 & 0 & 0 & 0 \\
    \end{bmatrix}
    \text{ and }
    \mathbf{B} =
    \begin{bmatrix}
        0 & 0 \\
        0 & \frac{L_1}{J_p} \\
        0 & 0 \\
        \frac{L_3}{J_e} & 0\\
        0 & 0 \\
        0 & 0 \\
    \end{bmatrix}
\end{gather}
And this gives us the linearized equations:
\begin{subequations} \label{eq:lin_model}
    \begin{align}
        &\tilde{\ddot{p}} = K_1 \tilde{V_d} \label{eq:lin_model_pitch}\\
        &\tilde{\ddot{e}} = K_2 \tilde{V_s} \label{eq:lin_model_elev}\\
        &\tilde{\ddot{\lambda}} = K_3 \tilde{p} \label{eq:lin_model_travel}
    \end{align}
\end{subequations}
where
\begin{gather}
    K_1 = \frac{L_1}{J_p}, K_2 = \frac{L_3}{J_e} \text{ and } K_3 = -\frac{L_2 L_4}{L_3 J_\lambda}
\end{gather}

\subsection{Problem 3}
?????

\subsection{Problem 4}
Now we want to determine the motor voltage constant, $K_f$, by measuring the value of $V_s = V_s^*$ which makes the helicopter maintain the equilibrium value $e = e^* = 0$. We measured the voltage to be $V_s = V_s^* = 6.75 \text{ V}$. 
\\From equation \eqref{eq:vs_tilde}, we have that $V_s^* = -\frac{L_2}{L_3}$. Furthermore, we can expand this expression using equation \eqref{eq:constant-L2} and equation \eqref{eq:constant-L3}, then solve for $K_f$ and finally insert the constant values and the measured value for $V_s^*$. 
\begin{gather}
    V_s^* = -\frac{L_2}{L_3} = \frac{g(2 l_h m_p - l_c m_c)}{l_h K_f}\nonumber\\
    \implies K_f = \frac{g(2 l_h m_p - l_c m_c)}{l_h V_s^*} = 0.148
\end{gather}
\section{Part II - Monovariable control}
\subsection{Problem 1 - Pitch angle PD controller}
A PD controller is to be added to control the pitch angle $p$. This controller is given as:
\begin{gather}
    \tilde{V}_d = K_{pp}(\tilde{p}_c-\tilde{p})-K_{pd}{\dot{\tilde{p}}}\label{eq:pd-controller}
\end{gather}
From equation \eqref{eq:lin_model_pitch} we see that $\tilde{V}_d = \frac{\ddot{\tilde{p}}}{K_1}.$ Substituting this into the equation for the PD controller \eqref{eq:pd-controller} gives:
\begin{gather*}
    \tilde{\ddot{p}} = K_1(K_{pp}(\tilde{p}_c - \tilde{p}) - K_{pd}\tilde{\dot{p}} \\
    \ddot{\tilde{p}} + K_1K_{pd}\dot{\tilde{p}} + K_1K_{pp}\tilde{p} = K_1 K_{pp} \tilde{p}_c
\end{gather*}
Using the Laplace transform and assuming $\tilde{p}(0) = \dot{\tilde{p}}(0) = 0$, gives us the following transfer function:
\begin{gather}
    s^2\tilde{p}(s) + sK_1K_{pd}\tilde{p}(s) + K_1K_{pp}\tilde{p}(s) = K_1K_{pp}\tilde{p}_c(s)\nonumber \\
    \tilde{p}(s)(s^2 + sK_1K_{pd} + K_1K_{pp}) = K_1K_{pp}\tilde{p}_c(s)\nonumber \\
    \frac{\tilde{p}(s)}{\tilde{p}_c(s)} = \frac{K_1K_{pp}}{s^2 + K_1K_{pd}s + K_1K_{pp}}
    \label{eq:transfer-func}
\end{gather}
The linearized pitch dynamics can be regarded as a second-order linear system, generally given by the transfer function:
\begin{gather*}
    h(s) = \frac{K\omega_0^2}{s^2 + 2\zeta\omega_0s + \omega_0^2}
\end{gather*}
We wish to obtain critical damping, in order to get the system to return to its equilibrium in the minimum amount of time without overshooting. Hence, we set $\zeta = 1$. The general transfer function for this type of system is given by:
\begin{gather}
    h(s) = \frac{K\omega_0^2}{s^2 + 2\omega_0s + \omega_0^2}
    \label{eq:gen-transfer-func}
\end{gather}
Comparing the transfer function of our system, equation \eqref{eq:transfer-func}, to the general one, equation \eqref{eq:gen-transfer-func}, gives the following equations for the regulator parameters (with K = 1):
\begin{gather}
    K_{pp} = \frac{\omega_0^2}{K_1} \text{ and } K_{pd} = 2 \frac{\omega_0}{K_1}
    \label{eq:gains}
\end{gather}
Now, we need to tune the controller gains $K_{pp}$ and $K_{pd}$. Our goal is to make the controller regulate the pitch angle rapidly to its desired value, without excessive oscillations. The natural frequency $\omega_o$ decides the rapidness of the pitch regulation. Increasing $\omega_o$ will make the controller react to pitch changes faster. This is because we move the poles of the system further into the left plane, seen by examining the the denominator of equation \eqref{eq:gen-transfer-func}:
\begin{gather*}
    (s + \omega_0)^2
\end{gather*}
 Looking at the expressions for  $K_{pp}$ and $K_{pd}$, given in equation \eqref{eq:gains}, we see that they're both dependant of $\omega_0$. Unsurprisingly, this means that tuning of the controller gains influences the closed-loop eigenvalues and the pitch response. A higher value for $K_{pp}$ for instance, will move the eigenvalues to the left in the complex plane, and give faster pitch response. However, there is a limit to how much the controller gains, and thus the natural frequency, may be increased. Increasing it too much may damage the helicopter, or make it unstable. 
 Through a process starting with initial guesstimates, followed by tuning by trial and error, we found values that gave us a satisfying response:
\begin{gather*}
    K_{pp} = 14 \\ 
    K_{pd} = 9.765
\end{gather*}
After including the PD controller for the pitch angle, we experienced that the helicopter was easier to control than it was when only feed forward was used. This was expected. The Matlab script for this problem can be found in \cref{subsec:P2_init.m}. The Simulink model can be found in \cref{sec:simulink} \cref{fig:P2p1} and \cref{fig:P2p1_PD}.

\subsection{Problem 2 - Travel rate P controller}
We now aim to control the travel rate $\dot{\tilde{\lambda}}$ using a simple P controller:
\begin{gather*}
    \tilde{p}_c = K_{rp}(\dot{\tilde{\lambda}}_c - \dot{\tilde{\lambda}})
    \label{eq:p-controller}
\end{gather*}
where $K_{rp} < 0$. We now assume that the pitch angle is controlled perfectly, that is $\tilde{p} = \tilde{p}_c$. Substituting equation \eqref{eq:lin_model_travel} into equation \eqref{eq:p-controller} for the P controller gives:
\begin{gather*}
    \ddot{\tilde{\lambda}} = K_3(K_{rp}(\dot{\tilde{\lambda}}_c - \dot{\tilde{\lambda}})) \\ 
    \ddot{\tilde{\lambda}} + K_3K_{rp}\dot{\tilde{\lambda}} = K_3K_{rp}\dot{\tilde{\lambda}}_c
\end{gather*}
Using the Laplace transform and assuming  $\tilde{\lambda}(0) = \dot{\tilde{\lambda}}(0) = 0$, gives us the transfer function from the reference $\dot{\tilde{\lambda}}_c$ to the travel rate $\dot{\tilde{\lambda}}$:
\begin{gather*}
    s\dot{\tilde{\lambda}}(s) + K_3K_{rp}\dot{\tilde{\lambda}}(s) = K_3K_{rp}\dot{\tilde{\lambda}}_c(s) \\
    \dot{\tilde{\lambda}}(s)(s + K_3K_{rp}) = K_3K_{rp}\dot{\tilde{\lambda}}_c(s) \\
    \frac{\dot{\tilde{\lambda}}(s)}{\dot{\tilde{\lambda}}_c(s)} = \frac{K_3K_{rp}}{s + K_3K_{rp}}
\end{gather*}
The transfer function can be written as:
\begin{gather*}
    \frac{ \dot{\tilde{\lambda}}(s)}{\dot{\tilde{\lambda}}_c(s)} = \frac{\rho}{s + \rho}
\end{gather*}
where $\rho = K_3K_{rp}$. We tested the helicopters response for different values for $K_{rp}$, and found that $K_{rp} = -1.2$ gave a fast and accurate response. The Matlab script for this problem can be found in \cref{subsec:P2_init.m}. The Simulink model can be found in \cref{sec:simulink} \cref{fig:P2p2} and \cref{fig:P2p2_P}. 
\section{Results and Figures}\label{sec:figures}
Answer all the parts of the exercise in an organized and clear manner. You should of course try to get good results in all the exercises, but if you have made a good effort without achieving great performance, a good discussion of possible reasons is just as good. Present your thinking and efforts and discuss possible reasons for good or bad results.

Include plots and/or tables of all relevant results, but make sure you don't overwhelm the reader with too many plots. Have a clear plan about what you want to communicate with a specific plot/figure, and use appropriate labels and comments. Keep in mind that the plots should be as ``readable'' as possible; that is, they should not be too hard to interpret and be reasonably self contained.

There are some important things to consider when exporting figures from MATLAB, most importantly which format you use. Never ever use JPEG for anything that is not a photography or similar. Any figure, like a plot or block diagram, must never be stored as a JPEG\@. If you zoom in on \Cref{fig:constraint_jpg} you can see a lot of noise close to any of the dark curves and lines, this is due to the compression in JPEG\@. \Cref{fig:constraint_jpg} will look horrible both on a screen and on paper.

The PNG format is slightly better for plots, but since it is a raster format (a grid of pixels), it looks ugly if you zoom in. It also looks ugly if you scale it, both on a screen and on paper. Try to avoid PNG if you can. \Cref{fig:constraint_png,fig:constraint_png_large} are both PNG figures; the latter being a larger figure scaled more than the former. Note both how choppy and ugly the blue curve is, and how the different sizes create inconsistent font sizes.

The simplest way to get a reasonably good looking plot is to save it as EPS in MATLAB\@. Do this by clicking ``File'' in the figure window, and the ``Save As\ldots''; choose ``EPS file (*.eps)'' in the ``Save as type:'' menu.\footnote{pdfLatex does not support EPS directly, but since we have loaded the \emph{epstodf} package, this is not a problem.} \Cref{fig:constraint_eps} shows a plot in EPS format. Since EPS is a vector format, the Figure can be scaled and still look good (but mind the font size!). If you zoom in you can see that the curve and the letters/numbers are smooth. A figure in vector format will usually look good both on a screen and on paper.

Note that the size of the actual figure window in MATLAB determines how large the exported figure is. Hence, if you enlarge the figure window before exporting, you will need to scale the figure by a larger factor in the report. This will lead to a tiny font in the figure. There are many better ways of exporting graphics from MATLAB, but they quickly become fairly involved. The above method of exporting to EPS will in most cases give nice figures.

You can write Latex in your MATLAB figures. The script used to create \Cref{fig:constraint_jpg,fig:constraint_eps} is included in \Cref{sec:plot_constraint_m}. Do not use a screen shot of a scope of figure in MATLAB in your report.

\begin{figure}[htb]
	\centering
		\includegraphics[width=0.8\textwidth]{figures/constraint_jpg.jpg}
	\caption{A plot in JPEG format --- a very bad idea.}
\label{fig:constraint_jpg}
\end{figure}

\begin{figure}[htb]
	\centering
		\includegraphics[width=0.8\textwidth]{figures/constraint_png.png}
	\caption{A plot in PNG format --- a bad idea.}
\label{fig:constraint_png}
\end{figure}

\begin{figure}[htb]
	\centering
		\includegraphics[width=0.8\textwidth]{figures/constraint_png_large.png}
	\caption{A plot in PNG format --- a bad idea. This figure is originally larger than the other PNG figure, but both are scaled to the same size.}
\label{fig:constraint_png_large}
\end{figure}



Remember to reference all figures in the text. Figures have a number and should be referenced by that number (again, always use dynamic references). They also tend to float around, meaning they generally don't appear where you ask them to in the text. This is fine, do not try to force a figure (or a table) to appear in a particular place. As long a you refer to it, it's easy to find. No figure should be included without being referenced in the text.

If you look at the source code for including figures, you can see that the optional option \verb+[htb]+ has been used. This tells Latex where you wish the figure to appear, in prioritized order. \verb+h+ means ``Here'', t means ``Top of this page'', b means ``Bottom of this page'', and p (not used here) means ``on a Page with only floats (such as figures and tables)''. Note that your wish might not be granted, and this is because Latex actually optimizes the placement of figures. If you start forcing figures to be in specific places, it often leads to really strange layout somewhere else in the document. 

Generally, let Latex handle the documentation layout. This is one of the main reasons to choose Latex over software such as Microsoft Word.

\subsection{Results and Discussion}
All problems should have their own discussion of results. 

Remember: all plots and results need a description, explanation, and discussion.


\section{Part III - Multivariable control}\label{sec:part3}
\subsection{Problem 1}
We put the system in a state-space formulation on the form:
\begin{gather*}
    \mathbf{\dot{x}} = \mathbf{Ax} + \mathbf{Bu}
\end{gather*}
where A and B are matrices, and our state and input vector are: 
\begin{gather*}
    \mathbf{x} = 
    \begin{bmatrix}
        \tilde{p}\\\tilde{\dot{p}}\\\tilde{\dot{e}}
        \end{bmatrix}
     \text{  and  } \mathbf{u} = 
    \begin{bmatrix} \tilde{V_s}\\\tilde{V_d} \end{bmatrix}
\end{gather*}
This gives us the following state-space model:
\begin{gather*}
     \begin{bmatrix}
        \tilde{\dot{p}}\\\tilde{\ddot{p}}\\\tilde{\ddot{e}}
    \end{bmatrix}
    \text{ + } \begin{bmatrix}
        0 & 1 & 0 \\
        0 & 0 & 0 \\
        0 & 0 & 0 \\ \end{bmatrix}
    \begin{bmatrix}
        \tilde{p}\\\tilde{\dot{p}}\\\tilde{\dot{e}}
    \end{bmatrix}
    \text{ = } \begin{bmatrix}
        0 & 0 \\
        0 & K_1 \\
        K_2 & 0 \\
    \end{bmatrix}
    \begin{bmatrix} \tilde{V_s}\\\tilde{V_d} \end{bmatrix}
\end{gather*}

\subsection{Problem 2}
We now aim to track the reference $\mathbf{r}
= \begin{bmatrix} \tilde{p_c,} & \tilde{\dot{e_c}}
\end{bmatrix}^T$ for the pitch angle $\tilde{p}$ and elevation rate $\dot{\tilde{e}}$. The reference values $\tilde{p}_c$ and $\dot{\tilde{e}}_c$ are given by the joystick, the x-axis and y-axis respectively. Firstly, we examine if the system is controllable. The controllability matrix is given by:
\begin{gather*}
    \mathcal{C} = 
    \begin{bmatrix} \mathbf{B} & \mathbf{AB} &
    \mathbf{A^2B} \end{bmatrix}
    = \begin{bmatrix}
    0 & 0 & 0 & K_1 & 0 & 0 \\
    0 & K_1 & 0 & 0 & 0 & 0 \\
    K_2 & 0 & 0 & 0 & 0 & 0
    \end{bmatrix}
\end{gather*}
We see that $\mathcal{C}$ has full rank, which means that the system is controllable. In order to actually control the system, we implement a controller of the form:
\begin{gather*}
    \mathbf{u} = \mathbf{Pr} - \mathbf{Kx}
\end{gather*}
In the controller, $\mathbf{r}$ is the reference given by the joystick output. The matrix $\mathbf{K}$ corresponds to the linear quadraric regulator (LQR) for which the control input $\mathbf{u} = -\mathbf{Kx}$ optimizes the cost function
\begin{gather*}
    \mathit{J} = \int_{0}^{\infty}(\mathbf{x}^T(t)
    \mathbf{Qx}(t) + \mathbf{u}^T(t)\mathbf{Ru}(t))
    \mathit{dt}
\end{gather*}
The matrix $\mathbf{K}$ is obtained by using the MATLAB command K = lqr(A,B,Q,R), where $\mathbf{Q}$ and $\mathbf{R}$ are weighting matrices. For simplicity, $\mathbf{Q}$ and $\mathbf{R}$ are diagonal. When deciding their elements we started with an initial guess, and tuned from there. We looked for values making the helicopter fast and accurate, and ended up with the following:
\begin{gather*}
    \mathbf{Q} = \begin{bmatrix}
    91.2 & 0 & 0 \\ 
    0 & 50 & 0 \\
    0 & 0 & 100 \end{bmatrix}\textbf{, } \mathbf{R} = \begin{bmatrix} 1 & 0 \\ 0 & 1 \end{bmatrix}
\end{gather*}
Our state-space model with the controller is
\begin{gather*}
     \mathbf{\dot{x}} = \mathbf{Ax} + \mathbf{B}(\mathbf{Pr} - \mathbf{Kx})
\end{gather*}
We want to obtain a solution such that  \( \mathbf{y}(t) \rightarrow \mathbf{r} \)  as t \(\rightarrow \infty\). When the output gets stable,  
\(\mathbf{\dot{x}} \rightarrow 0\). We get:
\begin{gather*}
    0 = \mathbf{Ax_\infty} + \mathbf{B}(\mathbf{Pr} - \mathbf{Kx_\infty)} \\
    (\mathbf{A} - \mathbf{BK})\mathbf{x_\infty} = -\mathbf{BPr} \\
    \mathbf{x_\infty} = (\mathbf{BK} - \mathbf{A})^{-1}
    \mathbf{BPr}
\end{gather*}
Substituting this into \(\mathbf{y_\infty} = \mathbf{Cx_\infty}\) yields:      
\begin{gather*}
    \mathbf{y_\infty} = 
    \begin{bmatrix} \mathbf{C}(\mathbf{BK} - \mathbf{A})^{-1} \mathbf{B}\end{bmatrix}\mathbf{Pr}
\end{gather*}
which gives:
\begin{gather*}
    \mathbf{P} = 
    \begin{bmatrix}\mathbf{C}(\mathbf{BK} - \mathbf{A})^{-1}\mathbf{B}\end{bmatrix}^{-1}
\end{gather*}
Our \textbf{K} and \textbf{P}:
\begin{gather*}
    \mathbf{K} = \begin{bmatrix}
    0 & 0 & 10 \\
    9.5499 & 9.0841 & 0 \end{bmatrix}\textbf{, } \mathbf{P} =
    \begin{bmatrix} 0 & 10 \\ 9.5499 & 0 \end{bmatrix}
\end{gather*}

KOMMENTERE KONTROLLERVALG

\subsection{Problem 3}
We now modify the controller to include an integral effect for the elevation rate and the pitch angle. Including an integral effect makes our controller a PI controller, and results in two additional states. We call the new states $\gamma$ and $\zeta$, and their differential equations are given by:
\begin{gather*}
    \dot{\gamma} = \tilde{p} - \tilde{p_c}\\
    \dot{\zeta} = \dot{\tilde{e}} - \dot{\tilde{e_c}}
\end{gather*}
The state vector and the input vector are now given by:
\begin{gather*}
    \mathbf{x} = \begin{bmatrix}
    \tilde{p} \\ \tilde{\dot{p}} \\
    \dot{\tilde{e}} \\ \gamma \\ \zeta \end{bmatrix}
    \text{ and } \mathbf{u} = \begin{bmatrix}
    \tilde{V_s} \\ \tilde{V_d} \end{bmatrix}
\end{gather*}

HER MÅ VI HA PLOTS HVOR VI SAMMENLIGNER HELIKOPTERETS RESPONS MED OG UTEN INTEGRALEFFEKT. EVT. NYE Q,R OG K MED FORKLARING. 










\section{Part IV - State Estimation}\label{sec:part4}

\subsection{Problem 1 - State-space formulation}\label{subsec:P4p1}
We wish to derive a state-space formulation from the system of linearized equations derived in \ref{subsec:P1p2} on the form

\begin{equation}\label{eq:state-space}
\begin{split}
    \mathbf{\dot{x}} &= \mathbf{Ax + Bu}\\
    \mathbf{y} &= \mathbf{Cx}
\end{split}
\end{equation}
where \textbf{A, B} and \textbf{C} are matrices defined below. The state vector, the input vector and the output vector are now given by:

\begin{align}\label{eq:states_part4}
    \textbf{x} &= \begin{bmatrix} \tilde{p} \\ \dot{\tilde{p}} \\ \tilde{e} \\ \dot{\tilde{e}} \\ \tilde{\lambda} \\ \dot{\tilde{\lambda}} \end{bmatrix}, &\textbf{u} &= \begin{bmatrix} \tilde{V_{s}} \\ \tilde{V_{d}} \end{bmatrix}, &\mathbf{y} &= \begin{bmatrix} \tilde{p} \\ \tilde{e} \\ \tilde{\lambda} \end{bmatrix}
\end{align}
From the output equation and the state vector, we can directly see that the matrix \textbf{C} is given by 
\begin{equation}
    \mathbf{C} = 
    \begin{bmatrix}
    1 & 0 & 0 & 0 & 0 & 0 \\
    0 & 0 & 1 & 0 & 0 & 0 \\
    0 & 0 & 0 & 0 & 1 & 0
    \end{bmatrix}
\end{equation}
Furthermore, using the equations \eqref{eq:lin_model_pitch}-\eqref{eq:lin_model_travel}, the state equation can be written as
\begin{equation}\nonumber
    \mathbf{\dot{x}} = 
    \begin{bmatrix} 
        \dot{\tilde{p}} \\ 
        \ddot{\tilde{p}} \\ 
        \dot{\tilde{e}} \\ 
        \ddot{\tilde{e}} \\ 
        \dot{\tilde{\lambda}}\\ 
        \ddot{\tilde{\lambda}} 
    \end{bmatrix} = 
    \begin{bmatrix} 
        x_{2}\\ 
        K_1\tilde{V_d}\\ 
        x_4\\ 
        K_2\tilde{V_s}\\ 
        x_6\\ 
        K_3 x_1 
    \end{bmatrix} = \mathbf{Ax + Bu}
\end{equation}
which gives us the matrices
\begin{align}\label{eq:state_matrices_P4p1}
    &\mathbf{A} = \begin{bmatrix}
        0 & 1 & 0 & 0 & 0 & 0 \\
        0 & 0 & 0 & 0 & 0 & 0 \\
        0 & 0 & 0 & 1 & 0 & 0 \\
        0 & 0 & 0 & 0 & 0 & 0 \\
        0 & 0 & 0 & 0 & 0 & 1 \\
        K_3 & 0 & 0 & 0 & 0 & 0
    \end{bmatrix}, 
    &\mathbf{B} = 
    \begin{bmatrix}
        0 & 0 \\
        0 & K_1 \\
        0 & 0 \\
        K_2 & 0 \\
        0 & 0 \\
        0 & 0 
    \end{bmatrix}
\end{align}

\subsection{Problem 2 - Linear observer}\label{subsec:P4p2}

In order to examine the observability of the system, we calculate the observabiliy matrix using
\begin{equation}\label{eq:obs_matrix}
    {\mathcal {O}}={\begin{bmatrix}C\\CA\\CA^{2}\\\vdots \\CA^{n-1}\end{bmatrix}}
\end{equation}
We know that the criteria for a system to be observable is that \eqref{eq:obs_matrix} has full rank. In our case, this is achieved after the two first computations in the observability matrix:
\begin{equation}\label{eq:obs_matrix_calc}
    {\mathcal {O}}_{2}=
    {\begin{bmatrix}
        1 & 0 & 0 & 0 & 0 & 0\\
        0 & 0 & 1 & 0 & 0 & 0\\
        0 & 0 & 0 & 0 & 1 & 0\\
        0 & 1 & 0 & 0 & 0 & 0\\
        0 & 0 & 0 & 1 & 0 & 0\\
        0 & 0 & 0 & 0 & 0 & 1
    \end{bmatrix}}
\end{equation}
Hence, our system is indeed observable.
\\Next, we wish to create a linear observer for the system of the form:
\begin{equation}\label{observer_system}
    \mathbf{\dot{\hat{x}} = A\hat{x} + Bu + L(y - C\hat{x})}
\end{equation}
where \textbf{L} is the observer gain matrix.
Because the system is observable, we have the opportunity to place the poles of the system where we want, in order to obtain an optimal response. We can do this by choosing an appropriate gain matrix \textbf{L}. The estimator $\mathbf{\hat{x}}$ has the same poles as $\mathbf{A - LC}$. Generally, we want the error dynamics to be faster than the system itself. In practice, this means that the poles of $\mathbf{A - LC}$ should be placed further into the left half plane than the poles of $\mathbf{A - BK}$, which represents the system dynamics. Abiding by the [HVEM FANT PÅ DETTE] principle, we place the estimator poles on a circular arc in the left half plane, shown in [VIS FIGUR!!]The radius of this arc is determined by a multiple of the distance to the most negative pole of the original system. MATLAB provides a simple function place() to compute the observer gain matrix \textbf{L}. MATLAB code for implementing this is shown in [LEGG IN KODE]. \\
\\Choosing a good gain matrix will be very important for this system, because although we want a fast estimator, the further into the left half plane the estimator poles are located, the more we will amplify noise in the measurements. In our project, we ended up placing the estimator poles on an arc with radius 13 times the distance of the most negative pole from the original system.\\
\\As in part 3, we tried tuning with different values for the matrices \textbf{Q} and \textbf{R} until the desired behavoiur was reached. We seemed however to get the best results with similar values as in part III, see [REFERER TIL Q, R, K]. With the computed gain matrix
\begin{equation}\nonumber
\mathbf{L} = \begin{bmatrix}
97,94 & 6,20 & -13,90\\
2476,40 & 318,65 & -725,98\\
-2,47 & 98,99 &	1,78\\
-114,14 & 2600,04 &	80,90\\
15,36 & 2,71 & 96,90\\
786,98 & -148,13 & 2412,86
\end{bmatrix}
\end{equation}

We were able to reach satisfying results with both controllers from part III, plots of the estimated values versus the real values for both controllers are shown below, as well as a simulink-diagram to show the implementation of the observer.
LEGG INN PLOTTS HER! REAL VS ESTIMATE PÅ BEGGE TO

SIMULINK AV OBSERVER!

\subsection{Problem 3 - Observer without pitch}\label{subsec:P4p3}
We aim to control our system by only measuring $\tilde{e}$ and $\tilde{\lambda}$. The output matrix \textbf{C} becomes:
\begin{equation}\label{eq:C_P4p3}
\mathbf{C} = 
    \begin{bmatrix}
    0 & 0 & 1 & 0 & 0 & 0 \\
    0 & 0 & 0 & 0 & 1 & 0 
    \end{bmatrix}
\end{equation}
Again, as in \ref{subsec:P4p2}, we use \eqref{eq:obs_matrix} to examine if the system is observable:
\begin{equation}\label{eq:obs_matrix_calc_last}
    \mathcal {O}=
    {\begin{bmatrix}
        0 & 0 & 1 & 0 & 0 & 0\\
        0 & 0 & 0 & 0 & 1 & 0\\
        0 & 0 & 0 & 1 & 0 & 0\\
        0 & 0 & 0 & 0 & 0 & 1\\
        0 & 0 & 0 & 0 & 0 & 0\\
        -K_3 & 0 & 0 & 0 & 0 & 0\\
        0 & 0 & 0 & 0 & 0 & 0\\
        0 & -K_3 & 0 & 0 & 0 & 0\\
        0 & 0 & 0 & 0 & 0 & 0\\
        0 & 0 & 0 & 0 & 0 & 0\\
        0 & 0 & 0 & 0 & 0 & 0\\
        0 & 0 & 0 & 0 & 0 & 0
    \end{bmatrix}}
\end{equation}
We can clearly see that \eqref{eq:obs_matrix_calc_last} has full rank, so the system is observable when measuring only $\tilde{e}$ and $\tilde{\lambda}$. However, if one only measures $\tilde{p}$ and $\tilde{e}$, this is not the case! The output matrix becomes:
\begin{equation}\nonumber
\mathbf{C} = 
    \begin{bmatrix}
    1 & 0 & 0 & 0 & 0 & 0 \\
    0 & 0 & 1 & 0 & 0 & 0 
    \end{bmatrix}
\end{equation},
which leads to the observability matrix
\begin{equation}\nonumber
    \mathcal {O}=
    {\begin{bmatrix}
        1 & 0 & 0 & 0 & 0 & 0\\
        0 & 0 & 1 & 0 & 0 & 0\\
        0 & 1 & 0 & 0 & 0 & 0\\
        0 & 0 & 0 & 1 & 0 & 0\\
        0 & 0 & 0 & 0 & 0 & 0\\
        0 & 0 & 0 & 0 & 0 & 0\\
        0 & 0 & 0 & 0 & 0 & 0\\
        0 & 0 & 0 & 0 & 0 & 0\\
        0 & 0 & 0 & 0 & 0 & 0\\
        0 & 0 & 0 & 0 & 0 & 0\\
        0 & 0 & 0 & 0 & 0 & 0\\
        0 & 0 & 0 & 0 & 0 & 0
    \end{bmatrix}}
\end{equation}
Now, the observability matrix has rank 4, which is not full rank. Hence, the system is not observable and it will be impossible to control with a state estimator. This is due to the fact that the pitch, $\tilde{p}$, can be found by differentiating $tilde{\lambda}$ twice and multiplying with a constant $K_3$. [SETT INN REFERANSE FRA OPPG1]. This is not the case for ...... 

\begin{equation}\nonumber
\mathbf{L} = \begin{bmatrix}
14,62 &	0,69 & -2,08\\
55,29 & 5,41 & -16,11\\
-0,13 & 14,51 & 0,48\\
-0,60 & 55,75 & 3,37\\
2,27 & -0,65 & 14,27\\
16,44 & -5,27 & 52,36
\end{bmatrix}
\end{equation}
\include{conclusion}
\addcontentsline{toc}{section}{Appendix} % Remove this if you don't want the appendix included in the table of contents.
\appendix

\section{MATLAB Code}\label{sec:matlab}
This section contains our MATLAB code, but only the relevant and changed parts for each problem. 

\subsection{Part IV - Problem 2}\label{subsec:P4p2_init.m}
\lstinputlisting{code/P4p2_latex.m}

\section{Simulink Diagrams}\label{sec:simulink}
This section should contain your Simulink diagrams. Just like the plots, these should be in vector format, like in \Cref{fig:simulink}. Make them tidy enough to understand.

\subsection{A Simulink Diagram}
\Cref{fig:simulink} shows a Simulink diagram. You can use the \texttt{print\_simulink.m} function, included in the source code repository for this document, to export a Simulink model to EPS\@.
\begin{figure}[htb]
	\centering
		\includegraphics[width = \textwidth]{figures/simulink_fra_mal.pdf}
	\caption{A Simulink diagram.}
\label{fig:simulink}
\end{figure}

% \input simply inserts the contents of the file, while \include forces a \newpage.
% See \input vs. \include: http://tex.stackexchange.com/questions/246/when-should-i-use-input-vs-include

% References
\newpage
\addcontentsline{toc}{section}{References}
\printbibliography{}
\label{sec:bibliography}

\end{document}
